\documentclass[9pt,usenames,dvipsnames]{beamer}

%\usetheme{Warsaw}
%\usetheme{Rochester}
%\usetheme{Dresden}
%\usetheme{Frankfurt}
%\usetheme{PaloAlto}
%\usetheme{Rochester}
\usetheme{Madrid} % I think this one works best
%\setbeamertemplate{page number in head/foot}[totalframenumber]
\setbeamertemplate{frametitle}[default][center]
\useoutertheme{infolines}
\usecolortheme[named=brown]{structure}

\usepackage[utf8]{inputenc}
\usepackage{amsmath}
\usepackage{amsfonts}
\usepackage{amssymb}
\usepackage{graphicx}
\usepackage{caption}
\usepackage{subcaption}
\usepackage{hyperref}
\hypersetup{colorlinks=true, linkcolor=blue, urlcolor=blue}

% Only a few macros are use here, so I'll just put them inline
\def\ket#1{|#1\rangle}

%\titlegraphic{\includegraphics[width=5cm]{CircuitLounge.jpg}}
\author{Mike Witt}
\title{Interpretations of Quantum Theory}
%\setbeamercovered{transparent} 
\setbeamertemplate{navigation symbols}{} 
\institute{Physics Cafe} 
%\date{} 
%\subject{} 
\begin{document}

\begin{frame}
\titlepage
\end{frame}

%\begin{frame}[t]{Outline}
%\tableofcontents%[part=1,pausesections]
%\end{frame}

\begin{frame}[t]{What needs to be interpreted?}
\begin{itemize}
\vspace{8pt}
\item 
We don't need to ``interpret'' (for example) Newton's 2nd law
\begin{quote}
\vspace{8pt}
$acceleration = force / mass$
\end{quote}
%\vspace{0pt}
\item
What's different about Schr\"odinger's equation, the equivalent
in quantum mechanics?
\begin{quote}
\vspace{8pt}
$i\hbar \frac{\partial}{\partial t}\psi(t) = \hat{H} \psi(t)$
\end{quote}
\end{itemize}

\vspace{2pt}
One difference is that we intuitively understand what 
{\it acceleration, force,} and {\it mass} are. These are things that we all
agree actually exist in the real world. We can locate them, measure them
and so on.
\vspace{12pt}\\
When it comes to the main player in Schr\"odinger's equation, $\psi$,
there is no general agreement on what it really is. We can't measure it in
any direct way. We can't even say that it's located in some specific place.
\vspace{12pt}\\
And, there are other issues ...
\end{frame}

\begin{frame}[t]{The main issues}
\begin{itemize}

\vspace{8pt}
\item 
The ontological status of the wave function
\vspace{6pt}\\
What is $\psi$?

\vspace{12pt}
\item
The locality debate
\vspace{6pt}\\
Does quantum theory imply reality has some kind of non-local aspect.

\vspace{12pt}
\item
The measurement problem
\vspace{6pt}\\
The measurement postulate appears to contradict the rest of the theory.
\end{itemize}
\end{frame}

\begin{frame}[t]{Some interpretations}
\begin{itemize}
\vspace{6pt}
\item The Textbook Interpretation
\vspace{6pt}
\item the ``No Interpretation'' Interpretation
\vspace{6pt}
\item The Copenhagen Interpretation
\vspace{6pt}
\item Everett
\vspace{6pt}
\item The Pilot Wave Interpretation (de Broglie / Bohm)
\vspace{6pt}
\item The Transactional Interpretation (Cramer / Kastner)
\vspace{6pt}
\item Wave function collapse theories
\end{itemize}
\end{frame}

\begin{frame}[t]{The Textbook Interpretation}
  \vspace{8pt}
  The wave function evolves according to Schr\"odinger's equation until
  a measurement happens. Then the wave function collapses into one of its
  basis states.
  \vspace{16pt}\\
  {\setlength\parindent{12pt}
    Q: What {\it exactly} distinguishes a measurement from other interactions?
    \vspace{8pt}\\
    A: You'll know one when you see it.
    \vspace{16pt}\\
    Q: So $\psi$ real, or just a mathematical construction?
    \vspace{8pt}\\
    A: This is a physics textbook, not a philosophy book.
  } 
\end{frame}

\begin{frame}[t]{No Interpretation}
  \vspace{12pt}
  We have a mathematical model, and we know how to use that model to
  predict the results of observations that we make in the real world.
  That's what a scientific theory consists of, and to ask for more
  is reaching beyond the realm of science.
  \vspace{8pt}\\
  You may think that quantum theory is ``strange'' or that it raises
  questions that don't come up in classical physics. But that's simply
  because your intuitions about how things ``should'' be is conditioned
  by the fact that you evolved as a macroscopic organism and not at
  the scale where quantum effects predominate.
  \vspace{8pt}\\
  Note: I believe that Sidney Coleman, in his famous 
  ``Quantum Mechanics in Your Face'' lecture, may be taking this position.
  (See the references section.)
\end{frame}

\begin{frame}[t]{The Copenhagen Interpretation}
  \vspace{12pt}
  Often, the way quantum mechanics is discussed in a typical textbook is
  called the ``Copenhagen interpretation.'' I don't believe this is correct.
  First of all, Niels Bohr (the main guy in Copenhagen) appeared
  to take a more-or-less positivist approach. Perhaps not unlike what I'm
  calling the ``no interpretation.''
  \vspace{8pt}\\
  Secondly, there was no real consensus on the philosophy of quantum mechanics
  among the other physicists associated with the Copenhagen group, such as
  Werner Heisenberg, Wolfgang Pauli, and others. They all had their own
  points of view.
  \vspace{8pt}\\
  Just to be clear, the first three viewpoints we've considered (Textbook,
  No Interpretation, and Copenhagen) should not really be considered as
  ``interpretations.'' They are, rather, heuristics for solving problems
  (in the textbook case) or philosophical positions which deny the 
  {\it need for} an interpretation. 

\end{frame}

\begin{frame}[t]{Everett's Interpretation}
\begin{itemize}
\item The status of the wave function
  \vspace{6pt}\\
  It seems to me that in Everett's approach $\psi$ is a mathematical
  abstraction. But one which is fairly closely associated with the world,
  as it really exists. Perhaps making its interpretation more straightforward.
  Apparently some others disagree with this statement, saying that in
  Everett the world is ``made out of'' the wave function. 
\vspace{6pt}\\
\item Locality / Non-locality
  \vspace{6pt}\\
  I don't believe Everett implies any sort of non-locality. I think
  it can be interpreted in a perfectly local way. Again, apparently
  not everyone agrees with this.
\vspace{6pt}\\
\item The measurement problem
  \vspace{6pt}\\
  This is the issue that Everett {\it clearly} addressed. A one line
  explanation of Everett's interpretation could simply be:
  {\bf Delete the measurement postulate!}
  \vspace{6pt}\\
  Everett has only unitary (linear) evolution ala Schr\"odinger's equation, 
  and all the interactions implied by that. Nothing special happens when we do
  something that we call a measurement. We (and our instruments) simply
  interact with the systems we are observing, typically causing us to become
  entangled with them.
  \vspace{6pt}\\
  Let's take a look at how this works ...
\end{itemize}
\end{frame}

\begin{frame}[t]{Everett's Interpretation}
Mike flips a quantum coin: $\ket{\mathrm{Coin}}\otimes\ket{\mathrm{Mike}}$
\begin{itemize}

\vspace{12pt}
\item A textbook measurement
    \vspace{12pt}\\
    \fbox{
    \begin{minipage}{0.43\textwidth}
    \center{\underline{Before}}
    \vspace{8pt}\\
    $
     \frac{1}{\sqrt{2}}\ket{\mathrm{H}} 
     + \frac{1}{\sqrt{2}}\ket{\mathrm{T}}
     \otimes \ket{\mathrm{What\;will\;it\;be?}}
    $
    \vspace{9pt}\\
    \end{minipage}
    }
    \fbox{
    \begin{minipage}{0.43\textwidth}
    \center{\underline{After}}
    \vspace{4pt}\\
    Either: $\ket{\mathrm{H}}\otimes\ket{\mathrm{I\;see\;Heads}}$
    \vspace{4pt}\\
    Or: $\;\,\ket{\mathrm{T}}\otimes\ket{\mathrm{I\;see\;Tails}}$
    \end{minipage}
    }

\vspace{16pt}
\item An Everett {\it interaction}
    \vspace{12pt}\\
    \fbox{
    \begin{minipage}{0.43\textwidth}
    \center{\underline{Before}}
    \vspace{8pt}\\
    $
     \frac{1}{\sqrt{2}}\ket{\mathrm{H}} 
     + \frac{1}{\sqrt{2}}\ket{\mathrm{T}}
     \otimes \ket{\mathrm{What\;will\;it\;be?}}
    $
    \vspace{4pt}
    \end{minipage}
    }
    \fbox{
    \begin{minipage}{0.43\textwidth}
    \center{\underline{After}}
    \vspace{8pt}\\
    $
     \frac{1}{\sqrt{2}}\ket{\textrm{\small H}} 
     \otimes \ket{\textrm{\small See heads}}
     + \frac{1}{\sqrt{2}}\ket{\textrm{\small T}}
     \otimes \ket{\textrm{\small See tails}}
    $
    \vspace{-2pt}
    \end{minipage}
    }
\end{itemize}
\end{frame}

\begin{frame}[t]{Everett's Interpretation}
\vspace{0pt}
{\bf Problems with Everett}
\begin{itemize}
\vspace{6pt}
\item It's incoherent
    \vspace{6pt}\\
    Some very smart people say this, or something close. I {\it think}
    what they are {\it really} saying is that they simply can't accept
    the physical reality of quantum superpositions. In order for Everett
    to make any sense, you need to believe that superpositions are not
    just mathematical tools for calculation, but that they reflect some
    aspect of reality.
\vspace{6pt}
\item There's a problem with probabilities
    \vspace{6pt}\\
    This comes in a number of different forms ...
\vspace{6pt}
\item The basis ambiguity problem
    \vspace{6pt}\\
    In my view, this is the {\it real} problem (or mystery :-) with Everett.
    Note that
    this is typically called the ``preferred basis problem.'' I don't like
    that terminology because it pre-supposes that there {\it is} a preferred
    basis.
    \vspace{6pt}\\
    (Try to verbalize a short description of this.)
\end{itemize}
\end{frame}

\begin{frame}[t]{The Pilot Wave Interpretation}
\begin{itemize}
  \item The status of the wave function
  \vspace{6pt}\\
    The wave function (or at least a part of it) is a real, physical entity
    which exerts an influence on the paths of particles.
  \vspace{6pt}
  \item Locality / Non-locality
  \vspace{6pt}\\
    This interpretation has what some people have called 
    ``non-locality with a vengeance,'' with the pilot wave exhibiting explicit
    action at a distance.
  \vspace{6pt}
  \item The measurement problem
  \vspace{6pt}\\
    No problem here. In this interpretation we have real particles, with
    real positions and real trajectories. When we make an observation of
    the position of a particle, we simply see what that position actually is.
\end{itemize}
\end{frame}

\begin{frame}[t]{The Pilot Wave Interpretation}
\vspace{0pt}
{\bf How the pilot wave works}
\vspace{8pt}\\
    Take the complex wave function $\psi(\vec{x},t)$ from Schrödinger's
    equation. $\vec{x}$ is the position vector of a particle and $t$ 
    is time. Like any complex function $\psi$ can be rewritten in exponential 
    form as $\psi(\vec{x},t)=Re^{iS}$, where $R$ and $S$
    are both real functions, and $S$ is the angle or phase in complex space.
    \vspace{6pt}\\
    Note: $S$ is a function of position and time, $S=S(\vec{x},t).$
    \vspace{6pt}\\
    We then replace Newton's 2nd law $a = F/m$ with the formula 
    $v = \nabla S / m$. So in other words the velocity of a particle is 
    equal to the gradient of the phase of the wave function (divided by 
    the particle's mass). 
    \vspace{6pt}\\
    The fact that the velocity of the particle (rather than its acceleration)
    depends on the pilot wave is why we say the wave exerts an ``influence''
    rather than a ''force'' on the particle. In fact, the trajectories
    of particles in the pilot wave interpretation look nothing like the way
    we picture the trajectories of particles in Newtonian dynamics.
\end{frame}

\begin{frame}[t]{The Pilot Wave Interpretation}
\vspace{0pt}
{\bf Problems with the pilot wave}
\begin{itemize}
\vspace{6pt}
\item Non-locality
\vspace{6pt}\\
    Some folks object to this interpretation simply because of the inherent
    non-locality, saying something along the lines of 
    ``it violates relativity.'' It's not clear how much force this objection
    has. Although pilot wave scenarios appear to go against the ``spirit''
    of (say) special relativity, all the physically verifiable events are the
    same ones that you get with the standard theories and calculations.
\vspace{6pt}
\item Special initial conditions
    \vspace{6pt}\\
    In order to match the standard predictions of quantum theory, you need
    to assume special initial conditions for all the particles' positions
    and velocities (based on the Born rule).
\vspace{6pt}
\item Extensibility to quantum field theory
    \vspace{6pt}\\
    A more serious question arising out of the non-local nature of the
    pilot wave is the issue of whether it can be extended to work with
    quantum field theory. I'm really not sure what the status of this is.

\end{itemize}
\end{frame}

\begin{frame}[t]{The Transactional Interpretation}
\begin{itemize}
\vspace{6pt}
\item Based on (or at least inspired by) the Wheeler and Feynman 
    ``time-symmetric'' theory of emitters and absorbers (which I'm not
    really familiar with).

\vspace{6pt}
\item There are forward-evolving waves $\psi$ and backward-evolving waves
      $\psi^*$. These are real physical entities.

\vspace{6pt}
\item A ``handshake'' needs to take place between the $\psi$ and $\psi^*$
      in order for a ``transaction'' to complete.

\vspace{6pt}
\item In Kastner's version of the theory, space-time is actually created
      out of these transactions.

\vspace{6pt}
\item This is the only interpretation I'm aware of which starts from
      electrodynamics. It clearly doesn't fall into any of the other
      interpretational categories, and is particularly interesting in that
      (at least in Kastner) it claims to be at a more fundamental level
      than space-time.

\end{itemize}
\end{frame}

\begin{frame}[t]{Wave function collapse theories}
\begin{itemize}
\vspace{6pt}
\item Collapse theories are actually alternate {\em theories} to standard
    quantum theory. They ultimately predict different results (although 
    viable collapse theories must predict the same results for any experiments
    which have actually been done.)

\vspace{6pt}
\item The collapse theories I've read about are all based on periodic, random
    collapses of the wave function. The parameters of these collapses have to
    be carefully constructed so as not to violate the results of previous
    experiments while leaving open the possibility of a future experiment which
    could decide between the new theory and the standard theory.

\vspace{6pt}
\item Examples are the GRW Theory (GianCarlo Ghirardi, Alberto Rimini, 
    Tulio Weber) which postulates that the wave function of a fundamental
    particle will experience a collapse on the average of once in
    $10^8$ years, and ``Relativistic Flashy GRW'' which 
    (I believe Tim Maudlin says) exhibits Lorentz invariance as well as
    passing John Bell's ``local beable'' test. (For more information, see
    Quantum Non-Locality and Relativity by Time Maudlin, Chapter 10 in
    the Third Edition only.)
    
    
\end{itemize}
\end{frame}

\begin{frame}[t]{References}
\vspace{6pt}
\begin{itemize}
\item \href{http://ucispace.lib.uci.edu/handle/10575/1302}
    {The Theory of the Universal Wave Function} by Hugh Everett\\
    Everett's thesis actually starts on page 9 of the pdf you can download
    from this page.

\vspace{8pt}
\item \href{https://www.goodreads.com/book/show/13688768-emergent-multiverse}
    {The Emergent Multiverse} by David Wallace\\
    This is currently the best up-to-date reference for Everett's
    interpretation.

\vspace{8pt}
\item \href{http://plato.stanford.edu/entries/qm-everett/\#7}
    {Stanford - Everett's Relative-State Formulation of Quantum Mechanics}\\
    \href{http://plato.stanford.edu/entries/qm-manyworlds/}
    {Stanford - Many-Worlds Interpretation of Quantum Mechanics}\\
    Two articles on Everett in Stanford's online Encyclopedia
    (better quality than Wikipedia).

\vspace{8pt}
\item \href{https://www.goodreads.com/book/show/1334044.The\_Quantum\_Theory\_of\_Motion}{The Quantum Theory of Motion} by Peter Holland\\
    This appears to me to be one pretty complete book on (Bohm's version of?)
    the pilot wave. But I've only looked it over a little bit, and there
    are a {\it lot} of much more recent works on this topic.
\end{itemize}
\end{frame}

\begin{frame}[t]{References}
\vspace{0pt}
\begin{itemize}

\item \href{https://www.barnesandnoble.com/w/quantum-non-locality-and-relativity-tim-maudlin/1100226576?ean=9781444331271}{Quantum Non-Locality and Relativity} by Time Maudlin\\
    \href{https://www.barnesandnoble.com/w/philosophy-of-physics-tim-maudlin/1128567493?ean=9780691183527}{Philosophy of Physics: Quantum Theory} by Tim Maudlin\\
    These two books by Tim Maudlin have {\it very} good discussions of the pilot wave, wave function collapse theories, and many other aspects of the philosophy of quantum theory. Not a good source on Everett, which Maudlin (mostly) dismisses as incoherent.

\vspace{8pt}
\item \href{https://www.goodreads.com/book/show/26362020-the-quantum-handshake?ac=1&from_search=true&qid=Tta3kWOxqK&rank=1}{The Quantum Handshake} by John Cramer

\vspace{0pt}
\item \href{https://www.goodreads.com/book/show/24509073-the-transactional-interpretation-of-quantum-mechanics?ac=1&from_search=true&qid=CrasunyOEg&rank=1}{The Transactional Interpretation of Quantum Mechanics} by Ruth Kastner

\vspace{0pt}
\item \href{https://transactionalinterpretation.org/}{Ruth Kastner's Blog}
\vspace{2pt}\\
    John Cramer originated the transactional interpretation and Ruth Kastner is its current leading proponent. 

\vspace{8pt}
\item \href{http://media.physics.harvard.edu/video/?id=SidneyColeman_QMIYF}{Sidney Coleman’s “Quantum Mechanics in Your Face” video}\\
    I highly recommend this lecture by the late Sidney Coleman, who has his own very interesting viewpoint on things. IMO it starts to get interesting about 20 or so minutes in.

\end{itemize}
\end{frame}

\end{document}
